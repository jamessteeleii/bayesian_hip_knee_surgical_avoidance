% Options for packages loaded elsewhere
% Options for packages loaded elsewhere
\PassOptionsToPackage{unicode}{hyperref}
\PassOptionsToPackage{hyphens}{url}
\PassOptionsToPackage{dvipsnames,svgnames,x11names}{xcolor}
%
\documentclass[
]{article}
\usepackage{xcolor}
\usepackage[left=1in,right=1in,top=1in,bottom=1in]{geometry}
\usepackage{amsmath,amssymb}
\setcounter{secnumdepth}{-\maxdimen} % remove section numbering
\usepackage{iftex}
\ifPDFTeX
  \usepackage[T1]{fontenc}
  \usepackage[utf8]{inputenc}
  \usepackage{textcomp} % provide euro and other symbols
\else % if luatex or xetex
  \usepackage{unicode-math} % this also loads fontspec
  \defaultfontfeatures{Scale=MatchLowercase}
  \defaultfontfeatures[\rmfamily]{Ligatures=TeX,Scale=1}
\fi
\usepackage{lmodern}
\ifPDFTeX\else
  % xetex/luatex font selection
  \setmainfont[]{Latin Modern Roman}
\fi
% Use upquote if available, for straight quotes in verbatim environments
\IfFileExists{upquote.sty}{\usepackage{upquote}}{}
\IfFileExists{microtype.sty}{% use microtype if available
  \usepackage[]{microtype}
  \UseMicrotypeSet[protrusion]{basicmath} % disable protrusion for tt fonts
}{}
\makeatletter
\@ifundefined{KOMAClassName}{% if non-KOMA class
  \IfFileExists{parskip.sty}{%
    \usepackage{parskip}
  }{% else
    \setlength{\parindent}{0pt}
    \setlength{\parskip}{6pt plus 2pt minus 1pt}}
}{% if KOMA class
  \KOMAoptions{parskip=half}}
\makeatother
% Make \paragraph and \subparagraph free-standing
\makeatletter
\ifx\paragraph\undefined\else
  \let\oldparagraph\paragraph
  \renewcommand{\paragraph}{
    \@ifstar
      \xxxParagraphStar
      \xxxParagraphNoStar
  }
  \newcommand{\xxxParagraphStar}[1]{\oldparagraph*{#1}\mbox{}}
  \newcommand{\xxxParagraphNoStar}[1]{\oldparagraph{#1}\mbox{}}
\fi
\ifx\subparagraph\undefined\else
  \let\oldsubparagraph\subparagraph
  \renewcommand{\subparagraph}{
    \@ifstar
      \xxxSubParagraphStar
      \xxxSubParagraphNoStar
  }
  \newcommand{\xxxSubParagraphStar}[1]{\oldsubparagraph*{#1}\mbox{}}
  \newcommand{\xxxSubParagraphNoStar}[1]{\oldsubparagraph{#1}\mbox{}}
\fi
\makeatother


\usepackage{longtable,booktabs,array}
\usepackage{calc} % for calculating minipage widths
% Correct order of tables after \paragraph or \subparagraph
\usepackage{etoolbox}
\makeatletter
\patchcmd\longtable{\par}{\if@noskipsec\mbox{}\fi\par}{}{}
\makeatother
% Allow footnotes in longtable head/foot
\IfFileExists{footnotehyper.sty}{\usepackage{footnotehyper}}{\usepackage{footnote}}
\makesavenoteenv{longtable}
\usepackage{graphicx}
\makeatletter
\newsavebox\pandoc@box
\newcommand*\pandocbounded[1]{% scales image to fit in text height/width
  \sbox\pandoc@box{#1}%
  \Gscale@div\@tempa{\textheight}{\dimexpr\ht\pandoc@box+\dp\pandoc@box\relax}%
  \Gscale@div\@tempb{\linewidth}{\wd\pandoc@box}%
  \ifdim\@tempb\p@<\@tempa\p@\let\@tempa\@tempb\fi% select the smaller of both
  \ifdim\@tempa\p@<\p@\scalebox{\@tempa}{\usebox\pandoc@box}%
  \else\usebox{\pandoc@box}%
  \fi%
}
% Set default figure placement to htbp
\def\fps@figure{htbp}
\makeatother


% definitions for citeproc citations
\NewDocumentCommand\citeproctext{}{}
\NewDocumentCommand\citeproc{mm}{%
  \begingroup\def\citeproctext{#2}\cite{#1}\endgroup}
\makeatletter
 % allow citations to break across lines
 \let\@cite@ofmt\@firstofone
 % avoid brackets around text for \cite:
 \def\@biblabel#1{}
 \def\@cite#1#2{{#1\if@tempswa , #2\fi}}
\makeatother
\newlength{\cslhangindent}
\setlength{\cslhangindent}{1.5em}
\newlength{\csllabelwidth}
\setlength{\csllabelwidth}{3em}
\newenvironment{CSLReferences}[2] % #1 hanging-indent, #2 entry-spacing
 {\begin{list}{}{%
  \setlength{\itemindent}{0pt}
  \setlength{\leftmargin}{0pt}
  \setlength{\parsep}{0pt}
  % turn on hanging indent if param 1 is 1
  \ifodd #1
   \setlength{\leftmargin}{\cslhangindent}
   \setlength{\itemindent}{-1\cslhangindent}
  \fi
  % set entry spacing
  \setlength{\itemsep}{#2\baselineskip}}}
 {\end{list}}
\usepackage{calc}
\newcommand{\CSLBlock}[1]{\hfill\break\parbox[t]{\linewidth}{\strut\ignorespaces#1\strut}}
\newcommand{\CSLLeftMargin}[1]{\parbox[t]{\csllabelwidth}{\strut#1\strut}}
\newcommand{\CSLRightInline}[1]{\parbox[t]{\linewidth - \csllabelwidth}{\strut#1\strut}}
\newcommand{\CSLIndent}[1]{\hspace{\cslhangindent}#1}



\setlength{\emergencystretch}{3em} % prevent overfull lines

\providecommand{\tightlist}{%
  \setlength{\itemsep}{0pt}\setlength{\parskip}{0pt}}



 


\usepackage[font=scriptsize]{caption}
\usepackage[noblocks]{authblk}
\renewcommand*{\Authsep}{, }
\renewcommand*{\Authand}{, }
\renewcommand*{\Authands}{, }
\renewcommand\Affilfont{\small}
\usepackage{etoolbox}
\AtEndEnvironment{abstract}{\noindent\textbf{Keywords:} \KeywordList \par}
\makeatletter
\@ifpackageloaded{caption}{}{\usepackage{caption}}
\AtBeginDocument{%
\ifdefined\contentsname
  \renewcommand*\contentsname{Table of contents}
\else
  \newcommand\contentsname{Table of contents}
\fi
\ifdefined\listfigurename
  \renewcommand*\listfigurename{List of Figures}
\else
  \newcommand\listfigurename{List of Figures}
\fi
\ifdefined\listtablename
  \renewcommand*\listtablename{List of Tables}
\else
  \newcommand\listtablename{List of Tables}
\fi
\ifdefined\figurename
  \renewcommand*\figurename{Figure}
\else
  \newcommand\figurename{Figure}
\fi
\ifdefined\tablename
  \renewcommand*\tablename{Table}
\else
  \newcommand\tablename{Table}
\fi
}
\@ifpackageloaded{float}{}{\usepackage{float}}
\floatstyle{ruled}
\@ifundefined{c@chapter}{\newfloat{codelisting}{h}{lop}}{\newfloat{codelisting}{h}{lop}[chapter]}
\floatname{codelisting}{Listing}
\newcommand*\listoflistings{\listof{codelisting}{List of Listings}}
\makeatother
\makeatletter
\makeatother
\makeatletter
\@ifpackageloaded{caption}{}{\usepackage{caption}}
\@ifpackageloaded{subcaption}{}{\usepackage{subcaption}}
\makeatother
\usepackage{bookmark}
\IfFileExists{xurl.sty}{\usepackage{xurl}}{} % add URL line breaks if available
\urlstyle{same}
\hypersetup{
  pdftitle={A Bayesian Analysis Using Informative Priors of Surgical Avoidance in Knee and Hip Osteoarthritis Patients Undergoing a Pilot Programme of Physiotherapy and Resistance Exercise Based Intervention},
  pdfauthor={Myles Moore; Brett Long; Dawn Aitken; Josh Petterwood; James Steele},
  pdfkeywords={TO ADD},
  colorlinks=true,
  linkcolor={blue},
  filecolor={Maroon},
  citecolor={Blue},
  urlcolor={Blue},
  pdfcreator={LaTeX via pandoc}}


\title{A Bayesian Analysis Using Informative Priors of Surgical
Avoidance in Knee and Hip Osteoarthritis Patients Undergoing a Pilot
Programme of Physiotherapy and Resistance Exercise Based
Intervention\thanks{Preprint, please cite as: Moore, M., Long, B.,
Aitken, D., Petterwood, J., and Steele, J. (2025). A Bayesian Analysis
Using Informative Priors of Surgical Avoidance in Knee and Hip
Osteoarthritis Patients Undergoing Physiotherapy and Resistance Exercise
Based Intervention. medrxiv DOI: TO ADD. Address for correspondence:
james@steele-research.com}}


\author[1,2]{Myles Moore}
\author[1,6]{Brett Long}
\author[3]{Dawn Aitken}
\author[4]{Josh Petterwood}
\author[1,4,5,6]{James Steele}

\affil[1]{Kieser Australia, Melbourne, VIC, Australia}
\affil[2]{Tasmanian Centre for Mental Health Service Innovation,
Tasmanian Health Services, Hobart, TAS, Australia}
\affil[3]{Menzies Institute for Medical Research, University of
Tasmania, Hobart, TAS, Australia}
\affil[4]{Petterwood Orthopaedics, Calvary Hospital, Hobart, TAS,
Australia}
\affil[4]{Steele Research Limited, Eastleigh, Hampshire, UK}
\affil[5]{MacroFactor, Stronger by Science Technologies LLC, Raleigh,
North Carolina, USA}
\affil[6]{School of Health and Biomedical Sciences, Royal Melbourne
Institute of Technology, Melbourne, VIC, Australia}

% Store the keywords (expanded by Pandoc here)
\def\KeywordList{TO ADD}

\date{December 12, 2025}
\begin{document}
\maketitle
\begin{abstract}
Background: Surgical intervention for knee and hip osteoarthritis (OA)
remains common, yet growing evidence suggests that structured
physiotherapy and resistance exercise programs may be associated with
delayed need for surgery. This study employs a Bayesian framework to
estimate the time to surgery following non-surgical interventions,
leveraging prior evidence through meta-analytic models to generate
informative priors. Methods: A two-stage Bayesian time-to-event analysis
was conducted. First, meta-analytic Bayesian discrete time proportional
hazards models were developed using data from existing studies examining
surgery rates following physiotherapy and exercise interventions. These
informed the priors used in a subsequent time-to-event analysis of
patient-level data collected through private health insurer-funded
physiotherapy and resistance training programs. Participants (N = 81)
who had been informed by their surgeon they would need surgery in the
next three years undertook physiotherapy and resistance exercise
programmes funded by their private health insurer. They were followed
for up to five years to observe whether and when joint replacement
surgery occurred. Posterior distributions were used to estimate survival
probabilities at the discrete time points followed up (6, 12, 36, and 60
months). Results: Informative priors derived from 12 prior studies were
incorporated into the survival model, improving estimation efficiency
given sparse event data in the observational dataset. During the
follow-up period there were 23 patients who underwent surgery. Estimated
probabilities of remaining surgery-free were 89\% {[}95\%CI: 87\%,
91\%{]} at 6 months, 80\% {[}95\%CI: 77\%, 83\%{]} at 12 months, 74\%
{[}95\%CI: 70\%, 78\%{]} at 36 months, and 69\% {[}95\%CI: 64\%, 74\%{]}
at 60 months. Results suggest that a substantial proportion of patients
who undergo non-surgical intervention may avoid or delay surgery for
several years. Conclusion: This Bayesian analysis, integrating prior
evidence and long-term follow-up data, suggests patients who had been
informed they would require surgery and who engage structured
physiotherapy and resistance-based exercise interventions may avoid that
surgery for several years. The use of informative priors enhanced model
stability and interpretability in the context of moderate event rates.
\end{abstract}


\section{Introduction}\label{introduction}

The Global Burden of Disease study has highlighted the growing
prevalence in Australia of knee and hip osteoarthritis
(OA)\textsuperscript{1}. This growing burden of places additional
pressure for individuals with moderate-to-severe knee or hip OA to
undergo surgery, despite contemporary guidelines recommending
non-surgical interventions be prioritised in these
conditions\textsuperscript{2}. As such, surgical intervention for knee
and hip OA remains common and presents a considerable cost to healthcare
providers\textsuperscript{3}. This demand for total knee and hip
replacement surgeries is expected to grow across many countries. Indeed,
in the United States, it is estimated that there will have been a growth
of 673\% for total knee replacements and 174\% for total hip
replacements from 2005 to 2030\textsuperscript{4}, and this growing
burden have also been predicted to also occur in the United Kingdom,
Canada, New Zealand, Australia and Sweden, despite results varying
between countries\textsuperscript{5--9}.

Contemporary guidelines from\textsuperscript{2} are now recommending
that non-surgical interventions, such as structured physiotherapy and
exercise programs, be prioritised for the management of knee and hip OA.
Indeed, several meta-analyses have shown that exercise-based
interventions, such as resistance training, has been shown to improve
pain, function, muscle mass and strength among individuals with knee or
hip OA {[}\textbf{ADD CITATIONS}{]}. Supervised progressive resistance
training programmes have also been shown to result in greater adherence
to treatment, function, pain and quality of life among those individuals
with knee or hip OA with compared to when they completed a home exercise
program {[}\textbf{ADD CITATIONS}{]}. Further, in support of
guidelines\textsuperscript{2}, growing evidence suggests that
non-surgical interventions such as structured physiotherapy and exercise
programs may be associated with delayed need for surgery, similar
clinical outcomes to surgical intervention, and subsequently a
cost-effective intervention for allowing for early healthcare
savings\textsuperscript{10,11}. However, despite growing evidence there
has yet to be any kind of systematic evidence synthesis regarding
surgery rates subsequent to non-surgical interventions such as
physiotherapy, exercise, and education.

This study reports the results of a pilot programme which involved the
delivery of a private health insurer-funded structured physiotherapy and
resistance training programme for patients with moderate-to-severe knee
or hip OA to undergo surgery who had been told they would require
surgery within the next 5 years by their surgeon. A Bayesian framework
was applied to estimate the time to surgery following non-surgical
interventions, leveraging prior evidence through meta-analytic models of
existing studies reporting time to surgery to generate informative
priors for analysis of the pilot data.

\section{Methods}\label{methods}

\section{Results}\label{results}

\section{Discussion}\label{discussion}

\section{Conclusion}\label{conclusion}

\section{Financial Disclosures/Conflicts of
Interest}\label{financial-disclosuresconflicts-of-interest}

ADD OTHERS

James Steele provides research consultancy through his company Steele
Research Limited, is contracted currently by MacroFactor and Kieser
Australia through Steele Research Limited, and has also received travel
expenses and honorarium for speaking from fit20 International, Exercise
School Portugal, and Discover Strength.

The content is solely the responsibility of the authors and does not
necessarily represent the official views of any funding agencies or
institutions noted above. The authors declare no other conflicts of
interest related to the submitted work.

\section{Data Availability}\label{data-availability}

All code utilised for data preparation, transformations, analyses,
plotting, and reporting are available in the corresponding GitHub
repository
\url{https://github.com/jamessteeleii/bayesian_hip_knee_surgical_avoidance}.

\section{Contributions}\label{contributions}

James Steele, Myles Moore, and Brett Long conceived the idea for the
project. All authors contributed to the design of the project and
methods. James Steele performed the data extraction, conducted the
statistical analyses, and produced the data visualisations. All authors
contributed to interpreting the results and drafting the initial
manuscript. All authors contributed to editing the manuscript. All
authors read and approved the final manuscript.

\phantomsection\label{refs}
\begin{CSLReferences}{0}{1}
\bibitem[\citeproctext]{ref-ackermanGlobalBurdenDisease2023}
\CSLLeftMargin{1. }%
\CSLRightInline{Ackerman IN, Buchbinder R, March L. Global {Burden} of
{Disease Study} 2019: An opportunity to understand the growing
prevalence and impact of hip, knee, hand and other osteoarthritis in
{Australia}. \emph{Internal Medicine Journal}. 2023;53(10):1875-1882.
doi:\href{https://doi.org/10.1111/imj.15933}{10.1111/imj.15933}}

\bibitem[\citeproctext]{ref-australiancommissiononsafetyandqualityinhealthcareOsteoarthritisKneeClinical2024}
\CSLLeftMargin{2. }%
\CSLRightInline{Australian Commission on Safety and Quality in Health
Care. \emph{Osteoarthritis of the {Knee Clinical Care Standard} (2024)
\textbar{} {Australian Commission} on {Safety} and {Quality} in {Health
Care}}.; 2024.}

\bibitem[\citeproctext]{ref-ackermanProjectedBurdenPrimary2019}
\CSLLeftMargin{3. }%
\CSLRightInline{Ackerman IN, Bohensky MA, Zomer E, et al. The projected
burden of primary total knee and hip replacement for osteoarthritis in
{Australia} to the year 2030. \emph{BMC Musculoskeletal Disorders}.
2019;20(1):90.
doi:\href{https://doi.org/10.1186/s12891-019-2411-9}{10.1186/s12891-019-2411-9}}

\bibitem[\citeproctext]{ref-kurtzProjectionsPrimaryRevision2007}
\CSLLeftMargin{4. }%
\CSLRightInline{Kurtz S, Ong K, Lau E, Mowat F, Halpern M. Projections
of primary and revision hip and knee arthroplasty in the {United States}
from 2005 to 2030. \emph{The Journal of Bone and Joint Surgery American
Volume}. 2007;89(4):780-785.
doi:\href{https://doi.org/10.2106/JBJS.F.00222}{10.2106/JBJS.F.00222}}

\bibitem[\citeproctext]{ref-pedersenTotalHipArthroplasty2005}
\CSLLeftMargin{5. }%
\CSLRightInline{Pedersen AB, Johnsen SP, Overgaard S, Søballe K,
Sørensen HT, Lucht U. Total hip arthroplasty in {Denmark}: Incidence of
primary operations and revisions during 1996-2002 and estimated future
demands. \emph{Acta Orthopaedica}. 2005;76(2):182-189.
doi:\href{https://doi.org/10.1080/00016470510030553}{10.1080/00016470510030553}}

\bibitem[\citeproctext]{ref-hooperCurrentTrendsProjections2014}
\CSLLeftMargin{6. }%
\CSLRightInline{Hooper G, Lee AJ-J, Rothwell A, Frampton C.
\href{https://www.ncbi.nlm.nih.gov/pubmed/25225759}{Current trends and
projections in the utilisation rates of hip and knee replacement in {New
Zealand} from 2001 to 2026}. \emph{The New Zealand Medical Journal}.
2014;127(1401):82-93.}

\bibitem[\citeproctext]{ref-nemesProjectionsTotalHip2014}
\CSLLeftMargin{7. }%
\CSLRightInline{Nemes S, Gordon M, Rogmark C, Rolfson O. Projections of
total hip replacement in {Sweden} from 2013 to 2030. \emph{Acta
Orthopaedica}. 2014;85(3):238-243.
doi:\href{https://doi.org/10.3109/17453674.2014.913224}{10.3109/17453674.2014.913224}}

\bibitem[\citeproctext]{ref-cullifordFutureProjectionsTotal2015}
\CSLLeftMargin{8. }%
\CSLRightInline{Culliford D, Maskell J, Judge A, et al. Future
projections of total hip and knee arthroplasty in the {UK}: Results from
the {UK Clinical Practice Research Datalink}. \emph{Osteoarthritis and
Cartilage}. 2015;23(4):594-600.
doi:\href{https://doi.org/10.1016/j.joca.2014.12.022}{10.1016/j.joca.2014.12.022}}

\bibitem[\citeproctext]{ref-sharifProjectingDirectCost2015}
\CSLLeftMargin{9. }%
\CSLRightInline{Sharif B, Kopec J, Bansback N, et al. Projecting the
direct cost burden of osteoarthritis in {Canada} using a microsimulation
model. \emph{Osteoarthritis and Cartilage}. 2015;23(10):1654-1663.
doi:\href{https://doi.org/10.1016/j.joca.2015.05.029}{10.1016/j.joca.2015.05.029}}

\bibitem[\citeproctext]{ref-ackermanImplementingNationalFirstline2020}
\CSLLeftMargin{10. }%
\CSLRightInline{Ackerman IN, Skou ST, Roos EM, et al. Implementing a
national first-line management program for moderate-severe knee
osteoarthritis in {Australia}: {A} budget impact analysis focusing on
knee replacement avoidance. \emph{Osteoarthritis and Cartilage Open}.
2020;2(3):100070.
doi:\href{https://doi.org/10.1016/j.ocarto.2020.100070}{10.1016/j.ocarto.2020.100070}}

\bibitem[\citeproctext]{ref-dockingLifetimeCostEffectivenessStructured2024}
\CSLLeftMargin{11. }%
\CSLRightInline{Docking S, Ademi Z, Barton C, et al. Lifetime
{Cost-Effectiveness} of {Structured Education} and {Exercise Therapy}
for {Knee Osteoarthritis} in {Australia}. \emph{JAMA Network Open}.
2024;7(10):e2436715.
doi:\href{https://doi.org/10.1001/jamanetworkopen.2024.36715}{10.1001/jamanetworkopen.2024.36715}}

\end{CSLReferences}




\end{document}
